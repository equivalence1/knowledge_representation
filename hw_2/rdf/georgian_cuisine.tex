\documentclass{article}
\usepackage[utf8]{inputenc}
\usepackage[english, russian]{babel}

\usepackage{minted}

\begin{document}

\title{Онтология грузинской кухни}
\author{Кравченко Дима, СПб ВШЭ, 5й курс}

\maketitle

Сразу замечу, что, насколько я понял, в этом задании хочется, чтобы мы сами определяли
все классы/property и т.д., а не использовали те, что уже определены во всяких общедоступных
онтологиях типа dbpedia/ontology.

Для начала, мы определяем общий список префиксов, которые используем при построении онтологии.

Следующим шагов мы должны определить terminological knowledge (извиняюсь, я уже
не помню, как это нормально переводится на русский язык). Т.к. эта часть призвана
помочь нам в будущем записывать assertional knowledge (аналогично), которая состоит
из описания конкретных рецептов, то логично, что она будет содержать довольно много
определений, которые не связаны напрямую с грузинской кухней, но является общей для
описания рецептов в целом.

Затем мы определяем assertional knowledge, которое уже описывает конкретные рецепты блюд
грузинской кухни. Всего я определил 2 рецепта -- хинкалей и ципленка табака, но остальные
блюда задаются аналогично.

Я постарался снабдить свое описание подробными комментариями.

\begin{minted}{sparql}

# Префиксы

@prefix rdf:  <http://www.w3.org/1999/02/22-rdf-syntax-ns#> .
@prefix rdfs: <http://www.w3.org/2000/01/rdf-schema#> .
@prefix xmls: <http://www.w3.org/2001/XMLSchema#> .
@prefix :     <http://example.org/GeorgianCuisine#> .



# Terminological Knowledge

## Классы

### общее понятие кухни и национальной кухни
:Cuisine           rdf:type           rdfs:Class .

:NationalCuisine   rdf:type           rdfs:Class ;
                   rdfs:subClassOf    :Cuisine .

### общее понятие еды 
:Food              rdf:type           rdfs:Class .

### общее понятие рецепта
:Recipe            rdf:type           rdfs:Class .

### общее понятие ингредиента
:Ingredient        rdf:type           rdfs:Class .

### список ингредиентов
:IngredientList    rdf:type           rdfs:Class ;
                   rdfs:subClassOf    rdf:Seq .

### общее понятие шага приготовления
:CookingStep       rdf:type           rdfs:Class .

### список шагов приготовления
:CookingSteps      rdf:type           rdfs:Class ;
                   rdsf:subClassOf    rdf:Seq .

## Свойства

### свойство принадлежности рецепта кухне
:cuisine           rdf:type           rdfs:Property ;
                   rdfs:domain        :Recipe ;
                   rdfs:range         :Cuisine .

### свойство наличия у рецепта списка ингредиентов
:ing_list          rdf:type           rdfs:Property ;
                   rdfs:domain        :Recipe ;
                   rdfs:range         :IngredientList .

### ингредиент ссылается на еду
:ing_food          rdf:type           rdfs:Property ;
                   rdfs:domain        :Ingredient ;
                   rdfs:range         :Food .

### ингредиент ссылается на количество еды.
### т.к. что такое "количество" -- не очень понятно
### (мы в штуках/граммах/кг/ложках/etc.), то для простоты
### мы храним это просто как строку
:ing_quantity      rdf:type           rdfs:Property ;
                   rdfs:domain        :Ingredient ;
                   rdfs:range         xmls:string .

### элемент в списке ингредиентов
:ing_li            rdf:type           rdfs:Property ;
                   rdfs:subClassOf    rdf:li ;
                   rdfs:domain        :IngredientList ;
                   rdfs:range         :Ingredient .

### свойства наличия у рецепта списка шагов приготовления
:cooking_steps     rdf:type           rdfs:Property ;
                   rdfs:domain        :Recipe ;
                   rdfs:range         :CookingSteps .

### описание шага в свободной форме
:cs_desc           rdf:type           rdfs:Property ;
                   rdfs:domain        :CookingStep ;
                   rdfs:range         xmls:string .

### Помимо описания можно было бы еще добавить ссылку на картику
### но я не хочу переписывать все эти ссылки.

### элемент в списке шагов приготовления
:step_li           rdf:type           rdfs:Property ;
                   rdfs:subClassOf    rdf:li ;
                   rdfs:domain        :CookingSteps ;
                   rdfs:range         :CookingStep .

# Assertional Knowledge

### В самом начале нам надо определить сам объект грузинской кухни
:GeorgianCuisine rdf:type :NationalCuisine .


## Рецепт хинкалей

### см (https://eda.ru/recepty/osnovnye-blyuda/hinkali-26016)

### первое, что мы должны определить -- это объекты еды,
### которые нам будут встречаться в ингредиентах этого блюда

:Flour     rdf:type       :Food ;
           rdfs:label     "Мука" .

:Water     rdf:type       :Food ;
           rdfs:label     "Вода" .

:Salt      rdf:type       :Food ;
           rdfs:label     "Соль" .

:Veal      rdf:type       :Food ;
           rdfs:label     "Телятина" .

:Beef      rdf:type       :Food ;
           rdfs:label     "Говядина" .

:Fat       rdf:type       :Food ;
           rdfs:label     "Сало" .

:Onion     rdf:type       :Food ;
           rdfs:label     "Лук репчатый" .

:Garlic    rdf:type       :Food ;
           rdfs:label     "Чеснок" .

:Cumin     rdf:type       :Food ;
           rdfs:label     "Молотый кумин" .

:Chilli    rdf:type       :Food ;
           rdfs:label     "Перец чили" .

:Cilantro  rdf:type       :Food ;
           rdfs:label     "Кинза" .

### список ингредиентов
:KhinkaliIngredientList   rdf:type  :IngredientList ;
                          :ing_li   [ a              :Ingredient ;
                                      :ing_food      :Flour ;
                                      :ing_quantity  "500 г"^^xmls:string ] ;
                          :ing_li   [ a              :Ingredient ;
                                      :ing_food      :Water ;
                                      :ing_quantity  "300 мл"^^xmls:string ] ;
                          :ing_li   [ a              :Ingredient ;
                                      :ing_food      :Salt ;
                                      :ing_quantity  "1.5 чайные ложки"^^xmls:string ] ;
                          :ing_li   [ a              :Ingredient ;
                                      :ing_food      :Veal ;
                                      :ing_quantity  "250 г"^^xmls:string ] ;
                          :ing_li   [ a              :Ingredient ;
                                      :ing_food      :Beef ;
                                      :ing_quantity  "250 г"^^xmls:string ] ;
                          :ing_li   [ a              :Ingredient ;
                                      :ing_food      :Fat ;
                                      :ing_quantity  "100 г"^^xmls:string ] ;
                          :ing_li   [ a              :Ingredient ;
                                      :ing_food      :Onion ;
                                      :ing_quantity  "100 г"^^xmls:string ] ;
                          :ing_li   [ a              :Ingredient ;
                                      :ing_food      :Garlic ;
                                      :ing_quantity  "2 зубчика"^^xmls:string ] ;
                          :ing_li   [ a              :Ingredient ;
                                      :ing_food      :Cumin ;
                                      :ing_quantity  "по вкусу"^^xmls:string ] ;
                          :ing_li   [ a              :Ingredient ;
                                      :ing_food      :Chilli ;
                                      :ing_quantity  "по вкусу"^^xmls:string ] ;
                          :ing_li   [ a              :Ingredient ;
                                      :ing_food      :Cilantro ;
                                      :int_quantity  "по вкусу"^^xmls:string ] .

### шаги приготовления.
### Я не стал вставлять сюда текста самих шагов, т.к. они очень длинные.
:KhinkaliCookingSteps     rdf:type  :CookingSteps ;
                          :step_li  [ a              :CookingStep ;
                                      :cs_desc       "1."^^xmls:string ] ;
                          :step_li  [ a              :CookingStep ;
                                      :cs_desc       "2."^^xmls:string ] ;
                          :step_li  [ a              :CookingStep ;
                                      :cs_desc       "3."^^xmls:string ] ;
                          :step_li  [ a              :CookingStep ;
                                      :cs_desc       "4."^^xmls:string ] ;
                          :step_li  [ a              :CookingStep ;
                                      :cs_desc       "5."^^xmls:string ] .

### и, наконец, сам рецепт
:KhinkaliDish   rdf:type           :Recipe ;
                :ing_list          :KhinkaliIngredientList ;
                :cooking_steps     :KhinkaliCookingSteps ;
                :cuisine           :GeorgianCuisine .


## Рецепт цыпленка табака

### доопределяем те продукты, которые еще не определили
:MeltedButter         rdf:type           :Food ;
                      rdfs:label         "Топленое масло" .

:Chick                rdf:type           :Food ;
                      rdfs:label         "Цыпленок" .

:Pepper               rdf:type           :Food ;
                      rdfs:label         "Молотый перец" .

### список ингредиентов
:TChickenIngList          rdf:type    :IngredientList ;
                          :ing_li     [ a             :Ingredient ;
                                        :ing_food     :MeltedButter ;
                                        :ing_quantity "2 столовые ложки"^^xmls:string ] ;
                          :ing_li     [ a             :Ingredient ;
                                        :ing_food     :Chick ;
                                        :ing_quantity "1 штука"^^xmls:string ] ;
                          :ing_li     [ a             :Ingredient ;
                                        :ing_food     :Garlic ;
                                        :ing_quantity "1 головка"^^xmls:string ] ;
                          :ing_li     [ a             :Ingredient ;
                                        :ing_food     :Salt ;
                                        :ing_quantity "по вкусу"^^xmls:string ] ;
                          :ing_li     [ a             :Ingredient ;
                                        :ing_food     :Pepper ;
                                        :ing_quantity "по вкусу"^^xmls:string ] .

### шаги приготовления.
### Я не стал вставлять сюда текста самих шагов, т.к. они очень длинные.
:TChickenCookingSteps     rdf:type  :CookingSteps ;
                          :step_li    [ a              :CookingStep ;
                                        :cs_desc       "1."^^xmls:string ] ;
                          :step_li    [ a              :CookingStep ;
                                        :cs_desc       "2."^^xmls:string ] ;
                          :step_li    [ a              :CookingStep ;
                                        :cs_desc       "3."^^xmls:string ] ;
                          :step_li    [ a              :CookingStep ;
                                        :cs_desc       "4."^^xmls:string ] ;
                          :step_li    [ a              :CookingStep ;
                                        :cs_desc       "5."^^xmls:string ] .

### сам рецепт
:TChicken       rdf:type           :Recipe ;
                :ing_list          :TChickenIngList ;
                :cooking_steps     :TChickenCookingSteps ;
                :cuisine           :GeorgianCuisine .


\end{minted}
 
Понятно, что на сайте \texttt{eda.ru} у рецепта блюда намного больше всяких свойств, чем просто
список ингредиентов и пошаговая инструкция приготовления, но понятно, что, во-первых, эти
два все же являются основными, которых хватает именно для приготовления блюда, во-вторых,
все остальные свойства задаются аналогичным образом.

\end{document}